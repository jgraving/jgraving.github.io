%%%%%%%%%%%%%%%%% PREAMBLE %%%%%%%%%%%%%%%%%%%%%%%%%%%%
%Change the font size of your document - 10pt, 12.1pt, etc.
\documentclass[letterpaper,10pt,oneside]{article}
\usepackage[utf8]{inputenc}
\usepackage{setspace}
\usepackage{hyperref}
\usepackage{longtable}
\usepackage{fontawesome}

\usepackage{graphicx}

% set font for main text

%Change the margins to fit your CV/resume content
\usepackage[left=0.5in, right=0.5in, bottom=0.75in, top=0.75in]{geometry}

%Changes the page numbers - {arabic}=arabic numerals, {gobble}=no page numbers, {roman}=Roman numerals
\pagenumbering{gobble}

%%%%%%%%%%%%%%%%% END OF PREAMBLE %%%%%%%%%%%%%%%%%%%%%

\begin{document}


%%%%%%%%%%%%%%%%% NAME OF APPLICANT %%%%%%%%%%%%%%%%%%%

\LARGE{\textbf{Jacob M. Graving}}  \\
\normalsize

%%%%%%%%%%%%%%%%% CONTACT INFORMATION %%%%%%%%%%%%%%%%%


\begin{center}
\begin{tabular}{l l}
 Max Planck Institute of Animal Behavior		& \hspace{2in} \href{mailto:jgraving@gmail.com}{\faEnvelope{ } jgraving@gmail.com} \\
 University of Konstanz, Department of Biology	& \hspace{2in}  \\ %\href{tel:+4917620710858 }{\faPhone{ } +49 176 207 10858 }  \\
  Centre for the Advanced Study of Collective Behaviour    & \hspace{2in}  \href{http://jakegraving.com/}{\faGlobe{ } jakegraving.com}   \\
  Universit\"{a}tsstr. 10  & \hspace{2in}  \href{https://twitter.com/jgraving}{\faTwitter{ }  twitter.com/jgraving}   \\
   Konstanz, Germany 78464         & \hspace{2in} \href{https://github.com/jgraving}{\faGithub{ }  github.com/jgraving} \\
 
\end{tabular}
\end{center}

%%%%%%%%%%%%%%%%% MAIN BODY %%%%%%%%%%%%%%%%%%%%%%%%%%%
% The main body is contained in a tabular environment. To move sections onto the next page, simply end the tabular environment and begin a new tabular environment.
\begin{small}
\noindent \begin{longtable}{@{} l p{5.3in}l}

\large{\textbf{Research Interests}}
& Computational models for the study of animal behavior, Bayesian statistical inference, machine/deep learning, computer vision, probabilistic programming, nonlinear dynamics \vspace{2mm}\\ 

\Large{\textbf{Positions}} \vspace{5mm} \\
\large{2020--present} 
& \textbf{Research Scientist} \\
%& Focus: Collective Behavior
& {Max Planck Institute of Animal Behavior} \vspace{1mm}\\
%& {Department of Biology, University of Konstanz} \\
%& {Centre for the Advanced Study Collective Behaviour, University of Konstanz} \vspace{0.5mm} \\
& \textbf{Role:} Research Scientist in the Advanced Research Technology (ART) Unit tasked with developing novel, general-purpose methods for the study of animal behavior in laboratory and field environments using computer vision, machine/deep learning, and modern statistical techniques, such as Bayesian causal inference. \\

\Large{\textbf{Education}} \vspace{5mm} \\
 \large{2021} 
 & \textbf{Ph.D.  (Dr.rer.nat.), Biology} \\
 %& Focus: Collective Behavior \\

 & {Department of Collective Behaviour, Max Planck Institute of Animal Behavior} \\
  & {Department of Biology, University of Konstanz} \\
    & {Centre for the Advanced Study Collective Behaviour, University of Konstanz} \\
 	& {International Max Planck Research School (IMPRS) for Organismal Biology} \vspace{1mm} \\
     &\textbf{Advisor}: Prof. Dr. Iain D. Couzin\\
     &\textbf{Thesis Title}: ``Computer Vision and Deep Learning Methods for Measuring and Modeling Animal Behavior" \\
      & \textbf{Grade}: 0,0 (summa cum laude) \\
     &  \vspace{5mm}\\
     
 \large{2015} 
  & \textbf{M.S., Biology} \\
 %& Focus: Ethology, Neuroscience \\
     & {Department of Biological Sciences, Bowling Green State University}\vspace{1mm}  \\
         & \textbf{Advisor}: Prof. Daniel D. Wiegmann\\
    &\textbf{Thesis Title}: ``Nocturnal Homing in Amblypygids"
     & \vspace{5mm} \\
 \large{2013} 
  &\textbf{B.S., Biology} \\
% & Focus: Ethology, Neuroscience \\
      & {Department of Biological Sciences, Bowling Green State University} \\
    
     & \\
 \Large{\textbf{Publications}}  \vspace{5mm} \\
\large{In Revision}
& Bath, D.E., \textbf{Graving, J.M.}, Walter, T., Sridhar, V.H.,  Vizcaíno, J.P., Couzin, I.D. Collective detection and processing of distributed information by fish schools. In revision. \vspace{1mm} \\




%\large{In Review}
& \textbf{Graving, J.M.}, Couzin, I.D.  Probabilistic self-supervised deep learning reveals the structure of high-dimensional data. \vspace{1mm} \href{https://doi.org/10.1101/2020.07.17.207993}{bioR$\chi$iv: https://doi.org/10.1101/2020.07.17.207993} \\
% & \textbf{Graving, J.M.}*, Stahl, A.L.*, Stowasser, A., Bingman, V.P., Wiegmann, D.D., Buschbeck, E.K. Vision in the nocturnal amblypygid \textit{Phrynus marginemaculatus}.   \vspace{1mm}	\\
\large{2020}
& Li, L., Nagy, M., \textbf{Graving, J.M.}, Bak-Coleman, J., Guangming X., Couzin, I.D. (2020). Vortex phase matching as a strategy for schooling in robots and in fish. Nature Communications 11, 5408 \href{https://doi.org/10.1038/s41467-020-19086-0 }{https://doi.org/10.1038/s41467-020-19086-0} \vspace{1mm} \\
\large{2019}
 &\textbf{Graving, J.M.}, Chae, D., Naik, H., Li, L., Koger, B., Costelloe, B.R., Couzin, I.D. (2019). DeepPoseKit, a software toolkit for fast and robust animal pose estimation using deep learning. eLife, 8. \href{https://doi.org/10.7554/elife.47994}{https://doi.org/10.7554/elife.47994}\\ &\href{https://doi.org/10.1101/620245}{bioR$\chi$iv: https://doi.org/10.1101/620245} Code: \href{https://github.com/jgraving/deepposekit}{ https://github.com/jgraving/deepposekit} \\
 &Press: \href{https://www.quantamagazine.org/to-decode-the-brain-scientists-automate-the-study-of-behavior-20191210/}{Quanta Magazine}, 
 \href{https://doi.org/10.1038/s41592-019-0678-2}{Nature Methods},
 \href{https://doi.org/10.1038/d41586-019-02942-5}{Nature News \& Views},
 \href{https://elifesciences.org/digests/47994/machine-learning-animal-poses-to-understand-behavior}{eLife Science Digests}
 \vspace{1mm} \\
\large{2018}
& Alarc\'{o}n-Nieto, G.*, \textbf{Graving, J.M.}*, Klarevas-Irby, J.A.*, Maldonado-Chaparro, A.A., Mueller, I., and Farine, D.R. (2018) An automated barcode tracking system for behavioural studies in birds. Methods in Ecology and Evolution 9 (6), 1536-1547. \href{https://doi.org/10.1111/2041-210X.13005}{https://doi.org/10.1111/2041-210X.13005} \href{https://doi.org/10.1101/201590}{bioR$\chi$iv: https://doi.org/10.1101/201590}  \small{*contributed equally}  \vspace{1mm} \\
\large{2017}
& \textbf{Graving, J.M.}, Bingman, V.P., Hebets, E.A., and Wiegmann, D.D. (2017). Development of site fidelity in the nocturnal amblypygid \textit{Phrynus marginemaculatus}. Journal of Comparative Physiology A, 203(5), 313-328. \href{https://doi.org/10.1007/s00359-017-1169-5}{https://doi.org/10.1007/s00359-017-1169-5} \vspace{1mm} \\
& Bingman, V.P., \textbf{Graving, J.M.}, Hebets, E.A., and Wiegmann, D.D. (2017). Importance of the antenniform legs, but not vision, for homing by the neotropical whip spider \textit{Paraphrynus laevifrons}. Journal of Experimental Biology, 220(5), 885-890.  \href{https://doi.org/10.1242/jeb.149823}{https://doi.org/10.1242/jeb.149823}\\
&Press: \href{http://blogs.discovermagazine.com/inkfish/2017/01/24/whip-spiders-use-their-feet-to-smell-their-way-home}{Discover Magazine}, 
\href{https://www.youtube.com/watch?v=eUoFegXxmfo}{National Geographic} \vspace{1mm} \\

\large{2016}
& Wiegmann, D.D., Hebets, E.A., Gronenberg, W., \textbf{Graving, J.M.}, and Bingman, V.P. (2016). Amblypygids: model organisms for the study of arthropod navigation mechanisms in complex environments. Frontiers in Behavioral Neuroscience, 10, 47. \href{https://doi.org/10.3389/fnbeh.2016.00047}{https://doi.org/10.3389/fnbeh.2016.00047} \vspace{1mm} \\   

 & \\ 


 %\Large{\textbf{Research}}  \vspace{5mm} \\
 
 %\large{2015–2020}
%     & \textbf{Max Planck Institute of Animal Behavior,}\\
  %   	& \textbf{Department of Collective Behaviour} \\
    % & Iain D. Couzin \\
    % & ``Revealing the Behavioral Algorithms of Social Animals"  \vspace{1mm}\\
    % & \parbox{5.0in}{Studying how sensory information and internal state drive the collective dynamics of animal groups. Developing machine learning methods to collect and analyze behavioral data.}\\
    % & \\ 
 %\large{2011–2015}
   %  & \textbf{Bowling Green State University, Department of Biological Sciences} \\
    % & Daniel D. Wiegmann, Verner P. Bingman \\
    % & ``Navigation and Sensory Discrimination in Amblypygids" \vspace{1mm}\\
    % & \parbox{5.0in}{Studied how amblypygids, a group of nocturnal arachnids, navigate home in the dark}\\
    % & \\
 %\large{2013}
   %  & \textbf{Bowling Green State University, Department of Biological Sciences} \\
    % & Sheryl L. Coombs \\
    % & ``The Sensory Basis of Rheotaxis in Fish" \vspace{1mm}\\
    % & \parbox{5.0in}{Studied how fish use multimodal sensory information to orient to flow.}\\
    % & \\
 %\large{2009}
    % & \textbf{SETGO Summer Research Scholar, Bowling Green State University} \\
    % & Matthew L. Partin \\
    % & ``Phenotypic Plasticity in Photosynthetic Stony Corals" \vspace{1mm}\\
    % & \parbox{5.0in}{Studied how genetically identical coral propagules adapt their morphology and physiology to changing environments.}\\
    % & \\
\Large{\textbf{Teaching}}  \vspace{5mm} \\
 \large{2019}
	& \textbf{ASAB 2019 Summer Conference, University of Konstanz} \\
	& Workshop Organizer and Lecturer\\
	& – Seminar on "Machine Learning in the Behavioral Sciences" \\
	& – Practical Workshop on "Quantifying Behavior with Machine Learning" \\
	& \\
 \large{2016–2020}
    & \textbf{University of Konstanz, Department of Biology} \\
     & Lecturer and Project Advisor, Intensive Research Course for Master's Students \\
     & – Measuring Animal Behavior with Computer Vision \\
     & – Analyzing Behavioral Data \\
     & – Introduction to Programming in Python\\
     & \\
 \large{2013–2015}
 & \textbf{Department of Biological Sciences, Bowling Green State University} \\
 & Graduate Assistant \\
 & – Advanced Biostatistics\\
 & – Introduction to Biostatistics \\
 & – Population and Community Ecology \\
 & – Introductory Biology for Non-Science Majors \\
 & – Guest Lecture on ``Arthropod Navigation", Animal Behavior \\
 & \\
 %\large{2009-2012}
 %& \textbf{Bowling Green State University, Department of Biological Sciences} \\
 %& Student Coordinator and Teaching Assistant, Marine Biology Laboratory \\
 %& – Introduction to Inland Marine Research \\
 %& – Aquarium Husbandry \\
 %& – Reef Aquarium Husbandry I and II \\
 %& \\
 %\large{2009}
 %& \textbf{Bowling Green State University, Department of Environmental Science} \\
 %& Student Teaching Assistant, Introduction to Environmental Science \\
 %& \\
 
% \Large{\textbf{Funding}}  \vspace{5mm} \\
 %\large{2013–2015}
   %  & \textbf{Graduate Research Fellowship} \\
    % & 100\% Tuition Waiver and \$45,000 Stipend \\
    % & Bowling Green State University \\
    % & \\
 %\large{2013}
%& \textbf{Undergraduate Research Fellowship} \\
%& \$5000 Stipend, \$800 Research Funds \\
%& Bowling Green State University, Center for Undergraduate Research and Scholarship \\
%& \\
 %\large{2009–2011}
%& \textbf{T. Richard Fisher Biology Scholarship} \\
%& \$8000/year Tuition Scholarship \\
%& Bowling Green State University, Department of Biological Sciences \\
%& \\
 %\large{2009}
%& \textbf{Summer Research Fellowship} \\
%& \$5000 Stipend, \$1000 Research Funds \\
%& Science, Engineering, Technology Gateway Ohio (SETGO), National Science Foundation \\
%& \\
 %\large{2009–2013}
%& \textbf{Award of Scholars} \\
%& Merit-based 75\% Tuition Scholarship \\
%& Bowling Green State University, College of Arts and Sciences \\
%& \\
 \Large{\textbf{Invited Talks}}  \vspace{5mm} \\
 \large{2019}
 & \textbf{Revealing the Behavioral Algorithms of Social Animals} \\
 & Princeton Neuroscience Institute (PNI) \\
 & Princeton University, Princeton, New Jersey, USA \\
 & July 2, 2019 \\
 & \\
\large{2018}
& \textbf{Perception and Motion in Locust Swarms} \\
& Integrated Behavioral Research Group (IBRG) \\
& Princeton University, Princeton, New Jersey, USA \\
& March 16, 2018 \\
& \\
& \textbf{Perception and Motion in Locust Swarms} \\
& Department of Biological Sciences Seminar Series \\
& Bowling Green State University, Bowling Green, Ohio, USA \\
& February 28, 2018 \\
& \\
 \Large{\textbf{Outreach}}  \vspace{5mm} \\
\large{2017--2019}
& \textbf{Konstanzer Lange Nacht Der Wissenschaft} \\
& ``Long Night of Science" Public Outreach Event \\
& Volunteer \\
& Konstanz, Germany \\
& \\
\large{2016}
& \textbf{Das Schwarmverhalten der Fische} \\
& Public Seminar by Prof. Jens Krause \\
& Volunteer Co-organizer \\
& Konstanz, Germany \\
& \\
\large{2013–2014}
& \textbf{Kid's Tech University, Bowling Green State University} \\
& Public Outreach Event for Schoolchildren Grades K–8 \\
& Volunteer \\
& Bowling Green, Ohio, USA \\
& \\
%\large{2008–2010}
%& \textbf{The Toledo Zoo Aquarium} \\
%& Volunteer and Intern \\
%& Toledo, Ohio, USA \\
%& \\

 \Large{\textbf{Advisees}}  \vspace{5mm} \\
 \large{Graduate} 
 & Simon Gommel, M.S. Biology, University of Konstanz \\
 & Taylor Carter, M.S. Biology, University of Konstanz \\
 & Ingabritta Hormann, M.S. Biology, University of Konstanz \\

 & \\
 \large{Undergraduate}
 & Nicole Meister, B.S. Computer Science, Princeton University \\
  & Chiara Hirschkorn, B.S. Biology, University of Konstanz \\
  & Daniel Chae, B.S. Computer Science, Princeton University \\
 &  Connie Santangelo, B.S. Biology, Bowling Green State University \\
 & Lindsey Cunningham, B.S. Biology, Bowling Green State University \\
 & Tracy Togba, B.S. Biology, Bowling Green State University \\
& \\

 \Large{\textbf{Peer Review}}  \vspace{5mm} \\
 \normalsize{\textit{Journals: }} & eLife, Science Advances, PNAS, Methods in Ecology and Evolution \vspace{1mm} \\
 \normalsize{\textit{Grants: }} & IMPRS Project Grant, IMPRS Travel Grant \vspace{1mm} \\
 & \\

\Large{\textbf{Skills}} \vspace{5mm} \\
%\large{\textbf{Computational}} \\
\normalsize{\textit{Languages: }}
& Python (Expert), R (Intermediate) \vspace{1mm}\\ %\LaTeX \\
\normalsize{\textit{Applications: }} 
& Bayesian inference, statistical analysis, data visualization, \\ 
& machine learning, deep learning, computer vision, and image processing \vspace{1mm}\\
\normalsize{\textit{Libraries: }}
& Stan, TensorFlow, PyTorch, Numpyro, scikit-learn, OpenCV \\
%& \\
%\large{\textbf{Biological}} \\
%\normalsize{\textit{Physiology: }}
%& electrophysiology, histology, opthalmoscopy, fish lateral line disruption and visualization \vspace{1mm} \\
%\normalsize{\textit{Microscopy: }}
%& scanning and transmission electron microscopy, confocal, fluorescence, and general light microscopy \vspace{1mm}\\
%\normalsize{\textit{Field: }}
%& radio telemetry, photographic polarimetry \\
%& \\
\end{longtable}
\end{small}
%\newpage

%%%%%%%%%%%%%%%%% REFERENCES %%%%%%%%%%%%%%%%%%%%%%%%%%
% The reference section has links to your references' websites and email addresses.

%\noindent \begin{longtable}{@{} l p{3.5in}l p{3.5in}|}
% \Large{\textbf{References}} \vspace{5mm} \\
% & \href{http://www.collectivebehaviour.com/couzin}{\textbf{Iain D. Couzin}} & \href{https://www.bgsu.edu/arts-and-sciences/biological-sciences/faculty-and-staff/alphabetical-listing/verner-bingman.html}{\textbf{Verner P. Bingman}} \\
% & Director, Max Planck Institute of Animal Behavior &  Distinguished Research Professor  \\
% & Professor, University of Konstanz &  Bowling Green State University \\
% & Department of Collective Behaviour  & Department of Psychology \\
% & \small{\href{mailto:icouzin@ab.mpg.de}{icouzin@ab.mpg.de}} & \small{\href{mailto:vbingma@bgsu.edu}{vbingma@bgsu.edu}} \\
% &\small{+49 7531 88-4928} & \small{+1 (419) 372 6984} \\
%&& \\
 % & \href{https://www.bgsu.edu/arts-and-sciences/biological-sciences/faculty-and-staff/alphabetical-listing/daniel-wiegmann.html}{\textbf{Daniel D. Wiegmann}} & \href{https://www.bgsu.edu/arts-and-sciences/neuroscience/nmb-people/faculty/sheryl-coombs.html}{\textbf{Sheryl L. Coombs}} \\
 %& Associate Professor  &  Professor Emeritus \\
 %& Bowling Green State University & Bowling Green State University \\
 %& Department of Biological Sciences  & Department of Biological Sciences \\
% & \small{\href{mailto:ddwiegm@bgsu.edu}{ddwiegm@bgsu.edu}} & \small{\href{mailto:scoombs@bgsu.edu}{scoombs@bgsu.edu}} \\
% &\small{+1 (419) 372 2691} & \small{+1 (419) 372 1206} \\
 
 %&& \\
 %& \%href{https://www.bgsu.edu/arts-and-sciences/biological-sciences/faculty-and-staff/alphabetical-listing/robert-huber.html}{Robert Huber} \\
 %& Professor \\
 %& Bowling Green State University \\
 %& Department of Biological Sciences  \\
 %& \small{\href{mailto:rh.bgsu@gmail.com}{rh.bgsu@gmail.com}} \\
 %&\small{+1 (419) 372 7492} \\
%\end{longtable}

%\bibliography{publications}

\end{document}

