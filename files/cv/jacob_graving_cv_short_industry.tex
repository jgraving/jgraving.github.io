%%%%%%%%%%%%%%%%%%%%%%%%%%%%%%%%%%%%%%%%%%%%%%%%%%%%%%%%%%%%%%%%%
%%%%%%%%%%%%%%%%%%%%%%%%%%%%%%%%%%%%%%%%%%%%%%%%%%%%%%%%%%%%%%%%%
% PREAMBLE (Updated for modern, professional font)
%%%%%%%%%%%%%%%%%%%%%%%%%%%%%%%%%%%%%%%%%%%%%%%%%%%%%%%%%%%%%%%%%
\documentclass[10pt,letterpaper]{article}

% Encoding and fonts
\usepackage[utf8]{inputenc}
\usepackage[T1]{fontenc}
\usepackage[scaled]{helvet}
\renewcommand{\familydefault}{\sfdefault} % Use Helvetica as default font

% Page layout
\usepackage[margin=2cm]{geometry}
\setlength{\parindent}{0pt}    % No paragraph indentation
\setlength{\parskip}{6pt}      % Space between paragraphs

\usepackage{fancyhdr}
\pagestyle{fancy}
\fancyhf{} % clear all header and footer fields
% Set footer
\fancyfoot[L]{Jacob M. Graving}   % Left-aligned name
\fancyfoot[C]{}                   % Center (empty)
\fancyfoot[R]{\thepage}          % Right-aligned page number

\renewcommand{\headrulewidth}{0pt} % Remove header line
\renewcommand{\footrulewidth}{0pt} % Optional: add a line at the bottom
\pagenumbering{arabic}             % Turn page numbers back on

% Useful packages
\usepackage{hyperref}          % Clickable URLs
\usepackage{enumitem}          % Better lists
\usepackage{fontawesome}       % For GitHub, Envelope icons, etc.

% Make itemize spacing tighter
\setlist[itemize]{topsep=2pt, itemsep=2pt, leftmargin=*, parsep=0pt}

% Custom section heading
\usepackage{titlesec}
\titleformat{\section}{\large\bfseries}{}{0pt}{}[\titlerule]
\titlespacing{\section}{0pt}{1em}{0.5em}

%%%%%%%%%%%%%%%%%%%%%%%%%%%%%%%%%%%%%%%%%%%%%%%%%%%%%%%%%%%%%%%%%
% DOCUMENT START
%%%%%%%%%%%%%%%%%%%%%%%%%%%%%%%%%%%%%%%%%%%%%%%%%%%%%%%%%%%%%%%%%
\begin{document}
	
	%-------------------- HEADER -----------------------%
\begin{center}
	{\huge \textbf{Jacob M. Graving}}\\[6pt]
	\href{mailto:jgraving@gmail.com}{\faEnvelope\ jgraving@gmail.com}
	\quad | \quad
	\href{http://jakegraving.com/}{\faGlobe\ jakegraving.com}
	\quad | \quad
	\href{https://github.com/jgraving}{\faGithub\ github.com/jgraving}\\[8pt]
	
	% Use a smaller font size for the long affiliation line:
	{\small
		Max Planck Institute of Animal Behavior \quad $\cdot$ \quad
		University of Konstanz, Department of Biology \\
		Universit\"{a}tsstr. 10, Konstanz, Germany 78464
	}
\end{center}

%-------------------- SUMMARY -----------------------%
\section*{Summary}
Research scientist and machine learning engineer with extensive experience in \textit{statistical modeling}, \textit{Bayesian inference}, and scalable \textit{AI systems}. My work focuses on building interpretable models and reproducible pipelines to analyze high-dimensional, multimodal data—often in uncertain or noisy environments. I bring a strong record of leading cross-disciplinary research, mentoring teams, and translating complex data into interpretable, decision-relevant results. Experienced in collaborating with engineers and domain experts to deliver robust, deployable solutions grounded in scientific rigor.

%-------------------- SKILLS -----------------------%
\section*{Skills}
\textbf{Core strengths:} Bayesian modeling, probabilistic programming, uncertainty quantification, deep learning, causal inference, time-series analysis

\textbf{Tools:} PyMC, NumPyro, Stan, PyTorch, JAX, TensorFlow, Git, Jupyter, R

\begin{itemize}
	\item Built scalable pipelines for multimodal datasets and developed interpretable models for noisy real-world data
	\item Designed transformers and contrastive learning architectures for representation learning in behavioral time-series
	\item Mentored researchers on applied ML and open-source tools for behavioral science and neuroscience
	\item Proven ability to quickly adopt new technologies and apply them to real-world scientific problems
\end{itemize}






%-------------------- EXPERIENCE -----------------------%
\section*{Experience}
\textbf{2020--present}\\
\textbf{Research Scientist}\\
Advanced Research Technology Unit, Max Planck Institute of Animal Behavior
\begin{itemize}
	\item Lead independent research at the interface of machine learning and behavioral science, developing general-purpose tools to measure and model animal behavior in both lab and field environments.
	\item Design and maintain scalable ETL pipelines for multimodal behavioral datasets, ensuring data integrity and reproducibility across experimental systems.
	\item Collaborate with interdisciplinary teams—including software engineers, neuroscientists, and field biologists—to build AI-driven tools for behavior analysis and experimental optimization.
	\item Apply Bayesian inference, causal modeling, and uncertainty quantification to large-scale time series and experimental data to generate interpretable, data-driven insights.
	\item Provide technical guidance and mentorship across collaborative projects, promoting reproducible science and statistical rigor.
\end{itemize}

%-------------------- EDUCATION -----------------------%
\section*{Education}

\textbf{2021} \\
\textbf{Dr.rer.nat., Biology} (0.0 ‘summa cum laude’) \\
Max Planck Institute of Animal Behavior \& University of Konstanz — \textit{Germany} \\
International Max Planck Research School (IMPRS) \\
\textit{Thesis:} \href{http://nbn-resolving.de/urn:nbn:de:bsz:352-2-dgcbudqch6ix8}{Deep Learning and Computer Vision Methods for Measuring and Modeling Animal Behavior} \\
\newpage
\textbf{2015} \\
\textbf{M.Sc., Biology} — Bowling Green State University — \textit{USA} \\
\textit{Focus:} Animal Behavior, Neuroscience \\

\textbf{2013} \\
\textbf{B.Sc., Biology} — Bowling Green State University — \textit{USA}





	
	%-------------------- PUBLICATIONS -----------------------%
	\section*{Recent \& Selected Publications}
	\textbf{In Prep}\\
	\textbf{Graving, J.M.} and Foster, J.J. (in prep). Unwrapping Circular Statistics: Bayesian Linear Models for Circular Data.
	\begin{itemize}
		\item Introduces a novel Bayesian generalized linear model framework for circular data by incorporating the von Mises distribution.
		\item Jointly models mean direction and variance to improve interpretability of circular outcomes.
		\item Offers practical guidelines and implementations using Bayesian toolkits such as PyMC and Stan.
	\end{itemize}
	
	\textbf{Graving, J.M.}, Heins, C., Couzin, I.D. Revealing the Structure of Time-Series Data with Context Attraction-Repulsion Embeddings.
	\begin{itemize}
		\item Proposes a generalized contrastive learning framework for dimensionality reduction and visualization of time-series data using transformer-based sequence modeling.
		\item Demonstrates recovery of latent factors and interpretable structure from high-dimensional behavioral and synthetic datasets.
		\item Introduces a general tool for exploratory analysis and hypothesis generation in time-series modeling.
	\end{itemize}

	
	\vspace{4pt}
	\textbf{In Review}\\
	Bath, D.E., \textbf{Graving, J.M.}, Walter, T., Sridhar, V.H., Vizcaíno, J.P., Couzin, I.D. Collective detection and processing of distributed information by fish schools. In revision for \textit{Current Biology}. \\ \textit{Code}: \href{https://github.com/jgraving/bayesian_beta_regression}{github.com/jgraving/bayesian\_beta\_regression}
	\begin{itemize}
		\item Developed a Bayesian model to analyze collective information processing in fish schools.
		\item Modeled ~3 billion behavioral observations across 400+ conditions using JAX and NumPyro.
		\item Revealed how distributed sensing shapes group-level responses to environmental signals.
	\end{itemize}
	
	\vspace{4pt}
	\textbf{2025}\\
	Sayin, S., Couzin-Fuchs, E., Petelski, I., G\"unzel, Y., Salahshour, M., Lee, C.-Y., \textbf{Graving, J.M.}, Li, L., Deussen, O., Sword, G.A., Couzin, I.D. (2025). The behavioral mechanisms governing collective motion in swarming locusts. \textit{Science}, 387(6737), 995–1000. \href{https://doi.org/10.1126/science.adq7832}{https://doi.org/10.1126/science.adq7832} \\ \textit{Code}: \href{https://github.com/jgraving/sayin_locust_mixture_model}{https://github.com/jgraving/sayin\_locust\_mixture\_model}
	\begin{itemize}
		\item Led Bayesian modeling and statistical analysis of virtual reality (VR) behavioral data from locusts.
		\item Identified and solved a key methodological flaw in prior work—confounding of group coordination with group size.
		\item Helped validate the core findings and strengthen the paper's quantitative rigor.
	\end{itemize}
	
	\vspace{4pt}
	\textbf{2023}\\
	Koger, B., Deshpande, A., Kerby, J.T., \textbf{Graving, J.M.}, Costelloe, B. R., Couzin, I.D. (2023). Quantifying the movement, behaviour and environmental context of group‐living animals using drones and computer vision. \textit{Journal of Animal Ecology}. \href{https://doi.org/10.1111/1365-2656.13904}{doi:10.1111/1365-2656.13904} \textit{Code}: \href{https://github.com/benkoger/overhead-video-worked-examples}{github.com/benkoger/overhead-video-worked-examples}
	\begin{itemize}
		\item Co-developed a computer vision pipeline combining drone footage and deep learning to extract 3D behavioral and environmental data from animal groups in natural settings.
		\item Integrated spatial context reconstruction with fine-scale movement tracking to support large-scale behavioral ecology studies.
	\end{itemize}
	\newpage
	\vspace{4pt}
	\textbf{2020}\\
	Li, L., Nagy, M., \textbf{Graving, J.M.}, Bak-Coleman, J., Guangming X., Couzin, I.D. (2020). Vortex phase matching as a strategy for schooling in robots and in fish. \textit{Nature Communications}, 11, 5408. \href{https://doi.org/10.1038/s41467-020-19086-0}{doi:10.1038/s41467-020-19086-0}
	\begin{itemize}
		\item Analyzed high-resolution posture data of schooling fish to test energy-optimization strategies predicted by robotic models.
		\item Applied deep learning–based pose estimation and mechanistic modeling to quantify fluid-mediated interactions and coordinated motion patterns.
	\end{itemize}
	
	\vspace{4pt}
	\textbf{2019}\\
	\textbf{Graving, J.M.}, Chae, D., Naik, H., Li, L., Koger, B., Costelloe, B.R., Couzin, I.D. (2019). DeepPoseKit, a software toolkit for fast and robust animal pose estimation using deep learning. \textit{eLife}, 8. \href{https://doi.org/10.7554/elife.47994}{doi:10.7554/elife.47994} \\
	\textit{Press}: \href{https://www.quantamagazine.org/to-decode-the-brain-scientists-automate-the-study-of-behavior-20191210/}{Quanta Magazine}, 
	\href{https://doi.org/10.1038/s41592-019-0678-2}{Nature Methods},
	\href{https://doi.org/10.1038/d41586-019-02942-5}{Nature News \& Views},
	\href{https://elifesciences.org/digests/47994/machine-learning-animal-poses-to-understand-behavior}{eLife Science Digests} \\
	\textit{Code}: \href{https://github.com/jgraving/deepposekit}{github.com/jgraving/deepposekit}
	\begin{itemize}
		\item Created a general-purpose, few-shot deep learning framework for high-speed, high-accuracy animal pose tracking in Python/TensorFlow.
		\item Supervised development of key components including a custom annotation GUI and augmentation tools for low-data regimes.
	\end{itemize}
	
	%-------------------- TEACHING -----------------------%
	\section*{Teaching}
	\textbf{2023}\\
	Konstanz School of Collective Behavior, University of Konstanz
	\begin{itemize}
		\item Designed and led a workshop on “Probabilistic Machine Learning” for 30+ international PhD students, covering Bayesian inference, causal inference, and information theory in practice. \href{https://www.exc.uni-konstanz.de/kscb/}{https://www.exc.uni-konstanz.de/kscb/}
	\end{itemize}
	
	\vspace{4pt}
	\textbf{2022}\\
	Deep Learning Workshop, Max Planck Institute of Animal Behavior
	\begin{itemize}
		\item Led an applied workshop introducing PyTorch and deep learning fundamentals to 15 researchers.
	\end{itemize}
	
	\vspace{4pt}
	\textbf{2019}\\
	ASAB 2019 Summer Conference, University of Konstanz
	\begin{itemize}
		\item Gave an invited seminar on “Machine Learning in the Behavioral Sciences” to an audience of 400+.
		\item Co-organized a workshop on behavioral quantification and modeling using ML methods (27 participants).
	\end{itemize}
	
	\vspace{4pt}
	\textbf{2016--2020}\\
	University of Konstanz, Department of Biology
	\begin{itemize}
		\item Co-developed and taught an intensive research course on collective behavior for Master’s students.
		\item Supervised five Master’s thesis projects across behavior, computation, and data science.
		\item Delivered lectures and practicals on computer vision for behavior analysis, data processing, and Python programming.
	\end{itemize}
	
	%%-------------------- SKILLS -----------------------%
	%\section*{Skills}
	%\textbf{Programming Languages:} Python
	
	%\textbf{Core Expertise:} Bayesian inference, causal inference, statistical modeling, machine learning, deep learning, computer vision, image analysis, data visualization
	
	%\textbf{Libraries \& Frameworks:} PyTorch, JAX, NumPyro, Stan, TensorFlow, scikit-learn, OpenCV
	
	%\textbf{Tools:} Git, LaTeX, Jupyter
	
	%-------------------- (COMMENTED REFERENCES EXAMPLE) -----------------------%
	% \section*{References}
	% (Optional references section, currently commented out)
	% \begin{itemize}
	%     \item ...
	% \end{itemize}
	
\end{document}

%\newpage

%%%%%%%%%%%%%%%%% REFERENCES %%%%%%%%%%%%%%%%%%%%%%%%%%
% The reference section has links to your references' websites and email addresses.

%\noindent \begin{longtable}{@{} l p{2.5in}l p{2.5in}|}
 %\Large{\textbf{References}} \vspace{5mm} \\
 %& \href{https://www.ab.mpg.de/couzin}{\textbf{Iain D. Couzin}} & %\href{https://www.ab.mpg.de/wikelski}{\textbf{Martin Wikelski}} \\
 %& Director, Max Planck Institute of Animal Behavior &  Director, Max Planck Institute of Animal Behavior  \\
 %& Professor, University of Konstanz &  Professor, University of Konstanz \\
 %& Department of Collective Behaviour  & Department of Migration \\
 %& \small{\href{mailto:icouzin@ab.mpg.de}{icouzin@ab.mpg.de}} & %\small{\href{mailto:wikelski@ab.mpg.de}{wikelski@ab.mpg.de}} \\
 %&\small{+49 7531 88 4928} & \small{+49 7732 1501 25} \\
%&& \\
% & \href{http://www.biology.emory.edu/Berman/index.html}{\textbf{Gordon Berman}} & \href{https://www.cgmi.uni-konstanz.de/personen/prof-dr-oliver-deussen/}{\textbf{Oliver Deussen}} \\
%& Associate Professor  &  Professor \\
%& Emory University & University of Konstanz \\
%& Department of Biology & Visual Computing Group \\
% & \small{\href{mailto:gordon.berman@emory.edu}{gordon.berman@emory.edu}} & \small{\href{mailto:oliver.deussen@uni-konstanz.de}{oliver.deussen@uni-konstanz.de}} \\
% &\small{+1 404-727-0071} & \small{+49 (0) 7531 88-2778} \\

% && \\
%& \%href{https://www.bgsu.edu/arts-and-sciences/biological-sciences/faculty-and-staff/alphabetical-listing/robert-huber.html}{Robert Huber} \\
%& Professor \\
%& Bowling Green State University \\
%& Department of Biological Sciences  \\
%& \small{\href{mailto:rh.bgsu@gmail.com}{rh.bgsu@gmail.com}} \\
%&\small{+1 (419) 372 7492} \\
%\end{longtable}


%\end{document}
